% COMANDI FORMATTAZIONE PAGINA 
\usepackage{geometry} 
\usepackage{csquotes} 
\usepackage{graphicx} 
\usepackage[hidelinks]{hyperref}  
\usepackage{setspace} 
\usepackage{appendix} 
\usepackage{abstract}
\usepackage{tikz}
\usetikzlibrary{automata, positioning}
\usetikzlibrary{matrix,decorations.pathreplacing}
\usetikzlibrary{shapes.geometric, arrows, calc}
% LINEA SOPRA LE PAGINE
\usepackage{fancyhdr}
\pagestyle{fancy}
\fancyhf{}
\fancyhead[LE]{\leftmark}
\fancyhead[RO]{\rightmark}
\fancyfoot[LE,RO]{\thepage}

% COMANDI FORMATTAZIONE TESTO
\usepackage[utf8]{inputenc}
\usepackage{listings} 
\newcommand{\rcode}[1]{\texttt{\lstinline{#1}}}
\lstset{
  language=R,
  basicstyle=\ttfamily\small,
  keywordstyle=\color{black},
  commentstyle=\color{green!50!black},
  stringstyle=\color{red},
  stepnumber=1,
  breaklines=true
}  
\usepackage[svgnames]{xcolor} 
\usepackage{comment} 

% COMANDI ELEMENTI GRAFICI E IMMAGINI

% COMANDI FORMULE MATEMATICHE
\usepackage{amsmath}
\usepackage{amssymb}
\usepackage{physics}
\usepackage{placeins}
\usepackage{braket}
\usepackage{nicematrix}
% PER INSERIRE ESEMPI
\newtheorem{esempio}{Esempio}

% PER AVERE TESTI IN ITALIANO
\usepackage[italian]{babel}

% PRIMA PAGINA
\usepackage{frontespizio}
\usepackage[pages=some]{background}

% BIBLIOGRAFIA
\usepackage[style=numeric]{biblatex}
\setlength{\bibitemsep}{\baselineskip}
\addbibresource{biblio.bib}




