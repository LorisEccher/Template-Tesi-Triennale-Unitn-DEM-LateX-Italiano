# Istruzioni per l'uso del template

A) Come scrivere il testo: 
Essenzialmente, si scrive normalmente nelle varie cartelle. Tuttavia, ci sono due cose da tenere a mente:
- Un solo "invio" non fa andare il testo a capo nel PDF. Devi premere due volte il tasto "invio" per andare a capo.
- Per fare spazi personalizzati, usa il comando `\vspace{misura in cm}`. (Vedi Capitolo 2 per esempi).

B) Citazioni: 
Per citare un testo, devi prima inserirlo nel file `biblio.bib`, assegnargli una parola chiave che lo rappresenti (ad esempio, per un libro di Mario Rossi, lo puoi chiamare `MR`). Poi, usa il comando `\cite{MR}` per inserire la citazione nel testo. Il numero di riferimento verrà inserito automaticamente.  
(Vedi Capitolo 2 per un esempio pratico, ad esempio, la citazione dopo "ciaooo").

C) Formule matematiche:  
Per scrivere le formule matematiche, usa i seguenti comandi:
- `\begin{equation} ... \end{equation}` per una formula numerata.
- `\begin{equation*} ... \end{equation*}` per una formula non numerata.

Con l'asterisco, la formula non verrà numerata. Includi un `\label{parola chiave}` prima di `\end{equation}`, così puoi fare riferimento alla formula nel testo usando `\eqref{parola chiave}`, che ti fornirà la numerazione automatica.  
(Vedi Capitolo 1, Sezione 1 per esempi di formule numerate e Sezione 2 per esempi di formule non numerate).

D) Per fare esempi:  
Per scrivere esempi numerati, usa il seguente comando:
```latex
\begin{esempio}
    Testo dell'esempio
\end{esempio}

E) Nelle formule matematiche, i pedici devono essere scritti come $x_{j}$ e le potenze come $x^{j}$.

F) Se desideri utilizzare la notazione matematica all'interno del testo, usa i simboli del dollaro. La sintassi corretta è:

$Testo in matematica$

(Vedi il capitolo 2 per comprendere le differenze tra l'uso e il non uso del simbolo del dollaro).

G) I comandi \section{} aprono una sezione con il relativo titolo tra parentesi, mentre i comandi \subsection{} aprono una sottosezione con il relativo titolo. Infine, i comandi \subsubsection{} aprono una sottosottosezione. Un esempio è nel Capitolo 1. (Questi comandi saranno molto utilizzati!)

H) Gli spazi vuoti $ \ \ \ $ nelle formule matematiche sono utilizzati per distanziare a piacere gli elementi. Ad esempio, se voglio che il simbolo "=" sia più distante da "Y", scrivo: $Y \ \ \ = X$.

(ATTENZIONE! Non devi mettere nulla subito dopo il comando \ per distanziare, altrimenti LaTeX lo interpreterà come un comando).

I) I titoli dei capitoli devono essere inseriti in main.tex, subito dopo il comando \chapter{}.

J) Il titolo della tesi va inserito in inizio.tex, utilizzando il comando \Titolo{TITOLO DA METTERE}.

K) Le citazioni vanno inserite nel file biblio.bib in base al tipo di riferimento. Ogni citazione deve iniziare con il simbolo "@", seguito dal tipo di riferimento (ad esempio, @book, @article, @journal, ecc.). Non è necessario specificare subito se si tratta di un libro, un articolo, ecc., poiché il formato può essere modificato in seguito. L'importante è segnare correttamente la citazione nel file, così sarà pronta per essere utilizzata nel testo.


