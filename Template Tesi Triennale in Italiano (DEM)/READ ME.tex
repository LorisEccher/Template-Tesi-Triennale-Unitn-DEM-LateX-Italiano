A) COME SI SCRIVE IL TESTO: ALLORA ESSENZIALMENTE SI SCRIVE NORMALMENTE NELLA VARIE CARTELLE; TUTTAVIA CI SONO DUE COSETTE CHE SONO DA DIRE: UN SOLO TASTO "INVIO" NON FA SI CHE IL TESTO VADA A CAPO ANCHE NEL PDF BISOGNA CHE VI SIANO DUE SPAZI (DUE VOLTE INVIO). P PER FARE SPAZI PERSONALIZZATI USARE \vspace{misura in cm} (GUARDARE CAPITOLO 2)

B) PER CITARE UN QUALCOSA BISOGNA PRIMA RIPORTARLO IN BIBLIOGRAFIA (IN "biblio.bib"), DARGLI UNA PAROLA CHIAVE CHE LO RIFERISCA "es. libro di Mario Rossi lo denomino MR", POI USARE IL COMANDO "slash+cite{e mettere qui una parola chiave che lo rappresenta} \cite{eco2017come}
Verra fuori il numero corrispondente in automatico. (VEDI CAPITOLO 2 COME ESEMPIO: Dopo ciaooo c'è la citazione)

C) PER FARE FORMULE MATEMATICHE BISOGNA USARE LA SEGUENTE DICITURA:
\begin{equation}
    
\end{equation}
OPPURE 
\begin{equation*}
    
\end{equation*}
CON L'ASTERISCO NON VIENE NUMERATA, MENTRE CON L'ASTERISCO SI, INOLTRE E' BENE DARE UN RIFERIMENTO ALL'EQUAZIONE CON UNA PAROLA CHIAVE USANDO \label{Parola chiave} prima di \end{equation}, COSì CHE NEL TESTO POSSO CITARE LA FOMRULA CON IL COMANDO \eqref{Parola chiave} CHE DA LA NUMERAZIONE AUTOMATICA (VEDERE IN CAPITOLO 1 SEZIONE 1 PER LA NUMERATA E SEZIONE 2 PER LA NON NUMERATA)
(NELLA FORMULA ALLA SEZIONE 1 HO MESSO UN ESEMPIO DI SOMMATORIA COSì DA FAR VEDERE COME SI FA)

D) PER FARE ESEMPI USARE LA SEGUENTE DICITURA (Si numerano da soli, vedere capitolo uno alla sezione 2):
\begin{esempio}
    Testo dell'esempio
\end{esempio}


E) NELLE FORMULE MATEMATICHE I PEDICI VANNO MESSI CON x_{j} e le potenze con x^{j}

F) SE SI VUOLE USARE LA SCRITTURA MATEMATICA NEL TESO USARE I DOLLARI E METTERE LA DICITURA MATEMATICA $Testo in matematica$; (Vedere capitolo 2 per capire cosa succede con e senza dollaro)

G) I COMANDI \section{} APRONO LA SEZIONE CON IL RELATIVO TITOLO NELLE PARENTESI, I \subsection{} APRONO LE SOTTOSEZIONI CON I TITOLI NELLE PARENTESI, MENTRE \subsubsection{} APRONO LE SOTTO SOTTO SEZIONI. UN ESEMPIO è AL CAPITOLO 1. (Questi li userai molto!)

H) GLI $\ \ \ \ \ $ NELLE FORMULE MATEMATICHE SERVONO PER DISTANZIARE A PIACERE GLI ELEMENTI (AD ESEMPIO VIGLIO CHE IL = SIA PIù DISTANTE  DA Y ALLORA FACCIO $Y \ \ \ = X$ )
(ATTENZIONE! LO \ PER DISTANZIARE NON DEVE AVERE QUALCOSA NELLO SPAZIO IMMEDIATAMENTE SUCCESSIVO, SE NO LATEX PENSA CHE SIA UN COMANDO)

I) I TITOLI DEI CAPITOLI LI METTI IN MAIN.TEX DOPO \chapter{}

J) IL TITOLO DELLA TESI LO METTI IN inixio.tex in \Titolo{TITOLO DA METTERE}

K) LE CITAZIONI IN biblio.bib VANNO FATTE A SECONDA DI COSA TU STIA CITANDO, SONO TUTTE INTRODOTTE DA UNA @ E POI LO STESSO Latex TI PROPONE POSSIBILI VIE, SE @book, @article, @journal etc. NON IMPORTA TANTO METTERE DA SUBITO SE è UN LIBRO UN ARTICOLO... PERCè è SEMPRE MODIFICABILE ALLA FINE IL FORM, L'IMPORTANTE è CHE COMUNQUE LO SEGNI SUBITO NELLA CARTELLA COSì DA AVERLO PRONTO PER LE CITAZIONI NEL TESTO
